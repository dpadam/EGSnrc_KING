
%%%%%%%%%%%%%%%%%%%%%%%%%%%%%%%%%%%%%%%%%%%%%%%%%%%%%%%%%%%%%%%%%%%%%%%%%%%%%%%
%
%  EGSnrc manual: preface
%  Copyright (C) 2015 National Research Council Canada
%
%  This file is part of EGSnrc.
%
%  EGSnrc is free software: you can redistribute it and/or modify it under
%  the terms of the GNU Affero General Public License as published by the
%  Free Software Foundation, either version 3 of the License, or (at your
%  option) any later version.
%
%  EGSnrc is distributed in the hope that it will be useful, but WITHOUT ANY
%  WARRANTY; without even the implied warranty of MERCHANTABILITY or FITNESS
%  FOR A PARTICULAR PURPOSE.  See the GNU Affero General Public License for
%  more details.
%
%  You should have received a copy of the GNU Affero General Public License
%  along with EGSnrc. If not, see <http://www.gnu.org/licenses/>.
%
%%%%%%%%%%%%%%%%%%%%%%%%%%%%%%%%%%%%%%%%%%%%%%%%%%%%%%%%%%%%%%%%%%%%%%%%%%%%%%%
%
%  Author:          Iwan Kawrakow, 2003
%
%  Contributors:    Blake Walters
%                   Frederic Tessier
%                   Ernesto Mainegra-Hing
%
%%%%%%%%%%%%%%%%%%%%%%%%%%%%%%%%%%%%%%%%%%%%%%%%%%%%%%%%%%%%%%%%%%%%%%%%%%%%%%%


\newpage
\mbox{ }\vspace*{-35mm}\\
\section*{\begin{center}Preface\end{center}}
\addcontentsline{toc}{section}{\numberline{}Preface}
\index{preface}
% Replace commented line for the one with fixed date when commiting
% Beware: Using the macro below conflicts between CVS and latex!!!
% \lfoot[{\sffamily {\leftmark}}]{{\small Last edited $Date: 2013/01/04 15:08:41 $
\lfoot[{\sffamily {\leftmark}}]{{\small Last edited 2011/05/02 18:36:27
}}
\mbox{ }\vspace*{-5mm}\\
\noindent {\bfseries Sixth printing: May 2011}\\
A long time has passed since the last update to this document. 
Although no fundamental changes have been made to the system,
there have been a number of important additions and improvements
to the system since 2009. Numerous bugs have been corrected and
a new user code for free-air chamber (FAC) correction calculations 
was added to the distribution.
\begin{itemize} 
\item Use of arbitrary electron impact ionization (EII) 
cross section compilations. Added a data base of EII 
cross sections based on the DWBA/PWBA theory by 
Bote and Salvat.
\item For backward compatibility with previous calculations, users can 
now request the use of PEGS4 photon data.
\item Inclusion of C++ {\tt egs\_fac} application with example. 
This user-code implements a self-consistent algorithm for the 
fast calculation of FAC correction factors.
\item EII cross sections printout for materials in
  the simulation if user requests output of the photon cross-sections.
\item Updated several GUI's to include most of the latest additions
and corrected several bugs.
\end{itemize} 
% \vspace{3mm}\\
\noindent {\bfseries Fifth printing: July 2009}\\
There has been a huge number of changes in the system since 
the last printing in November 2003 that have not been properly 
added to the documentation. This printing is an attempt to update 
this report to better reflect the state of EGSnrc, although it 
is far from complete. The development of the EGSnrc C++ class 
library, which includes a general purpose geometry package, and its 
first public release in 2005 has been a major step forward. 
The C++ class library is described in a separate report (PIRS--898). 
Major changes and additions to the EGSnrc physics include: 
option to simulate electron impact ionization, an improved 
bremsstrahlung data base that includes an exact evaluation of 
electron-electron bremsstrahlung in the first Born approximation, 
an improved differential pair production cross section tabulation 
based on exact PWA calculations that takes into account the asymmetry 
of the energy distribution at energies close to the threshold, 
the ability to explicitely 
simulate triplet interactions (\ie, pair production in the electron field), 
the ability to take into account radiative corrections for Compton 
scattering in the one-loop approximation, the ability to use 
user-supplied atomic and molecular form factors for Rayleigh scattering, 
and the ability to use total photon cross sections from EPDL97, XCOM, or 
any other user-supplied tabulation in addition to the default 
Storm \& Israel tabulations. 
New options added for Compton scattering called {\tt simple} and 
{\tt norej}. When using {\tt simple} binding is taken
into account via an incoherent scattering function 
ignoring Doppler broadening. The incoherent scattering function 
is still obtained from the impulse approximation. 
The {\tt norej} option uses actual bound Compton cross section 
when initializing the photon cross sections and rejections in 
subroutine COMPT lead to re-sampling rather than rejecting the 
entire interaction.
An {\tt alias sampling} algorithm is now used to select the 
photon angle after a Rayleigh scattering to avoid an undersampling
at large angles observed in the original EGS4 implementation.
Particle track scoring object added to the {\tt egspp} library
allowing visualization of particle tracks with the C++ geometry
viewer {\tt egs\_view}. See example input file tracks1.egsinp for 
the tutor7pp C++ user-code.

Finally, to better reflect their contributions 
to the development and maintenence of the system, Ernesto Mainegra-Hing, 
Blake Walters and Frederic Tessier have been added as co-authors. 
\vspace{3mm}\\
\noindent {\bfseries Fourth printing: November 2003}\\
Added references to EGSnrcMP and Report PIRS-877. A few minor changes
related to the major change in the operating system to make it Windows
compliant. There is no associated change in the physics of the system.
Table 7 re timing of random numbers has changed substantially.
\vspace{3mm}\\
\noindent {\bfseries Third printing: April 2002}\\
Minor changes reflecting code changes. 
%See section~\ref{fixed_bugs}.
\vspace{3mm}\\
\noindent {\bfseries Second printing: May 2001}\\
The second printing contains a description of the use of {\tt RHOF}
(section~\ref{RHOF_RHOR}, page~\pageref{RHOF_RHOR}) and {\tt \$SET-RHOF}
(section~\ref{set_rhof}, page~\pageref{set_rhof}). There is a brief new
discussion of {\tt combine\_egsnrc}, a script for automatic analysis of the
many files created by {\t pprocess} in parallel runs
((section~\ref{pprocess}, page~\pageref{pprocess})). There is a new section
about terminating histories with WT=0.0 (section~\ref{termination},
page~\pageref{termination}). 
%Finally, a new section
%has been added which documents the few minor changes made to EGSnrc since
%its initial release ((section~\ref{fixed_bugs},
%page~\pageref{fixed_bugs})). 
\vspace{3mm}\\
\noindent {\bfseries First printing: May 2000}\\
In the decade and a half since the original version of EGS4 was released
there have been well over 1000 papers published which cite the original
SLAC-265 Report.  The code itself has been improved in many different ways
by a large number of people.  For a detailed history of much of this up to
1994, the reader is referred to a report titled  ``History, overview and
recent improvements of EGS4'' by Bielajew et al.
\index{Bielajew, Alex}

In the last few years there have been significant advances in several
aspects of electron transport.  For example, improvements in multiple scattering
theory have been developed by Kawrakow and Bielajew\cite{KB97,Bi96,Ka96} 
which get over most of
the shortcomings of the Moliere theory used in EGS4. Perhaps more important
has been the development by Kawrakow and Bielajew\cite{KB97a} of a new electron transport algorithm, sometimes
called PRESTA-II, which makes a significant advance in the science of
electron transport.  In addition to these advances, Kawrakow has implemented
several other improvements in the electron transport algorithm of EGS
which make it capable of accurately calculating ion chamber response at the
0.1\% level (relative to its own cross sections)\cite{Ka99a,Ka99b}.  

EGSnrc also has implemented a variety of additional features, many of
which have previously been extensively developed as additions to EGS4
by  Namito, Hirayama and Ban at KEK as well\cite{Na98,Na95a,Na94,Na93}.
The EGSnrc approach differs from that of the KEK group, partially because
once we were making fundamental changes to the code, we carried it through
in a consistent manner. However, the KEK group have implemented several
options which are not yet in EGSnrc (e.g.  polarized photon scattering
and electron impact ionization).
\index{Hirayama, Hideo} \index{Namito} \index{Ban}

This report is meant to document the many changes that have occurred going
from EGS4 to EGSnrc.  Although this report is written by two people,
the EGS system is obviously the child of many parents who have made a
wide variety of contributions over the years.  This goes right back to
Richard Ford, then at SLAC, who was a major contributor to EGS3. Hideo
Hirayama, S Ban  and Yosh Namito of KEK have made innumerable contributions
to EGS, especially concerning the low energy photon physics. Alex
Bielajew worked on EGS at NRC from the early 80's to late 90's and his
name is linked to a huge number of important contributions to EGS,
perhaps most importantly the PRESTA algorithms, but also many other
specific improvements to the physics, the Unix based scripts and the NRC
user codes. His name appears very extensively in the reference lists.
The name of Walter Ralph Nelson is practically synonymous with EGS and
all users of any version of the EGS system will forever be in Ralph's
debt. It has been his enthusiasm and willingness to help others and share
this resource so selflessly which has made it the great success it is.
To all of these people who have contributed so extensively to the EGS
system, and to the countless others who have played a variety of roles,
we all owe a huge debt of gratitude.
\index{Nelson, Ralph}


It is worth noting that NRC and SLAC have drawn up a formal agreement which
recognizes that both have rights associated with EGS4 and EGSnrc.  Thus, in
this report there are sections which are taken verbatim from SLAC-265 (in
particular the PEGS4 manual and the User's guide to Mortran3) and we wish
to thank SLAC for permission to reproduce them.  We also draw attention to
the copyright and licensing arrangements associated with EGSnrc which are
similar to those for EGS4, but which are becoming more tightly controlled
in this changing world we live in.  Neither EGS4 nor EGSnrc are public
domain software. They are both copyright protected by NRC and/or SLAC. The
formal license statement is more precise and part of the package, but the
general meaning is that individuals are granted a without cost 
license to use it for non-commercial purposes but that a license from NRC
is needed for any commercial application, and by definition someone working
for a for-profit organization or working on a contract for such an
organization is working on a commercial application.
\index{SLAC}


\noindent{\em What is next?\\}
In the section of the Preface to SLAC-265, there were 7 areas identified as
needing more work.  The work on EGS is not  complete, and at least 2 of
the 7 are still open, viz:
\begin{itemize} 
\vspace{-4mm}
\item development of an efficient, general purpose geometry package
tailored to the EGS structure
\vspace{-3mm}
\item implementation of a general purpose energy loss straggling algorithm
which properly handles energy cutoffs
\end{itemize} 

\vspace{-5mm}
There are other issues which are still undone within EGSnrc:
\vspace{-5mm}
\begin{itemize} 
\item modeling of electron impact ionization
\vspace{-3mm}
\item some critical feature for your next application!!
\end{itemize} 
We encourage users to contribute their improvements to the code. We will
happily add those which are of general interest and make available on the
distribution site those additions which are of special interest. We will
also appreciate receiving bug reports. Although we have done extensive QA
on the system, there have been many changes and not all parts of the code
are as carefully checked as we would like.  However, the pressure to release
the code is forcing us to proceed at this point.

We wish to thank our many colleagues at NRC who have helped with this work.
In particular Michel Proulx for his excellent help keeping the computer
systems going smoothly, Jan Seuntjens for his help with the user codes,
Joanne Treurniet for her help with the most recent version of the
EGS\_Windows system and Blake Walters for his work on QA of the system.

\noindent I.K and D.W.O.R.  \hfill Feb 2000 \vspace{1mm}\\

%\noindent {\bfseries Second printing: May 2001}\\
%The second printing contains a description of the use of {\tt RHOF}
%(section~\ref{RHOF_RHOR}, page~\pageref{RHOF_RHOR}) and {\tt \$SET-RHOF}
%(section~\ref{set_rhof}, page~\pageref{set_rhof}). There is a brief new
%discussion of {\tt combine\_egsnrc}, a script for automatic analysis of the
%many files created by {\t pprocess} in parallel runs
%((section~\ref{pprocess}, page~\pageref{pprocess})). There is a new section
%about terminating histories with WT=0.0 (section~\ref{termination},
%page~\pageref{termination}). Finally, a new section
%has been added which documents the few minor changes made to EGSnrc since
%its initial release ((section~\ref{fixed_bugs},
%page~\pageref{fixed_bugs})). \vspace{3mm}  \\
%\noindent {\bfseries Third printing: April 2002}\\
%Minor changes reflecting code changes. See section~\ref{fixed_bugs}.
%\vspace{3mm}\\
%\noindent {\bfseries Fourth printing: November 2003}\\
%Added references to EGSnrcMP and Report PIRS-877. A few minor changes
%related to the major change in the operating system to make it Windows
%compliant. There is no associated change in the physics of the system.
%Table 7 re timing of random numbers has changed substantially.
%\vspace{3mm}\\
%\noindent {\bfseries Fifth printing: July 2009}\\
%There has been a huge number of changes in the system since 
%the last printing in November 2003 that have not been properly 
%added to the documentation. This printing is an attempt to update 
%this report to better reflect the state of EGSnrc, although it 
%is far from complete. The development of the EGSnrc C++ class 
%library, which includes a general purpose geometry package, and its 
%first public release in 2005 has been a major step forward. 
%The C++ class library is described in a separate report (PIRS--898). 
%Major changes and additions to the EGSnrc physics include: 
%option to simulate electron impact ionization, an improved 
%bremsstrahlung data base that includes an exact evaluation of 
%electron-electron bremsstrahlung in the first Born approximation, 
%an improved differential pair production cross section tabulation 
%based on exact PWA calculations that takes into account the asymmetry 
%of the energy distribution at energies close to the threshold, the ability to explicitely 
%simulate triplet interactions (\ie, pair production in the electron field), 
%the ability to take into account radiative corrections for Compton 
%scattering in the one-loop approximation, the ability to use 
%user supplied atomic and molecular form factors for Rayleigh scattering, 
%and the ability to use total photon cross sections from EPDL97, XCOM, or 
%any other user-supplied tabulation in addition to the default 
%Storm \& Israel tabulations. Finally, to better reflect their contributions 
%to the development and maintenence of the system, Ernesto Mainegra-Hing, 
%Blake Walters and Frederic Tessier have been added as co-authors. 
