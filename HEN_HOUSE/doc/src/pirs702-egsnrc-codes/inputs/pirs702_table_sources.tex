
%%%%%%%%%%%%%%%%%%%%%%%%%%%%%%%%%%%%%%%%%%%%%%%%%%%%%%%%%%%%%%%%%%%%%%%%%%%%%%%
%
%  EGSnrc user codes description of source options
%  Copyright (C) 2015 National Research Council Canada
%
%  This file is part of EGSnrc.
%
%  EGSnrc is free software: you can redistribute it and/or modify it under
%  the terms of the GNU Affero General Public License as published by the
%  Free Software Foundation, either version 3 of the License, or (at your
%  option) any later version.
%
%  EGSnrc is distributed in the hope that it will be useful, but WITHOUT ANY
%  WARRANTY; without even the implied warranty of MERCHANTABILITY or FITNESS
%  FOR A PARTICULAR PURPOSE.  See the GNU Affero General Public License for
%  more details.
%
%  You should have received a copy of the GNU Affero General Public License
%  along with EGSnrc. If not, see <http://www.gnu.org/licenses/>.
%
%%%%%%%%%%%%%%%%%%%%%%%%%%%%%%%%%%%%%%%%%%%%%%%%%%%%%%%%%%%%%%%%%%%%%%%%%%%%%%%
%
%  Authors:         Iwan Kawrakow, 2003
%
%  Contributors:    Blake Walters
%
%%%%%%%%%%%%%%%%%%%%%%%%%%%%%%%%%%%%%%%%%%%%%%%%%%%%%%%%%%%%%%%%%%%%%%%%%%%%%%%


\begin{longtable}{lll}
\caption[Description of source options for the RZ codes]
{Description of the {\tt SOURCE OPTIONS=} inputs for the source 
routines in the RZ codes.  Additional inputs are also indicated
where required.  Delimeters are: {\tt :start source inputs:} 
and {\tt :stop source inputs:}.}\\
\hline\hline
Source & Parameters & Description \\
Number &&\\
\hline
\endfirsthead
\hline
\multicolumn{3}{r}{\small\slshape continued from previous page} \\
\hline \hline
Source & Parameters      & Description \\
Number &&\\
\hline
\endhead
\hline
\multicolumn{3}{r}{\small\slshape continued on next page} \\ \hline
\endfoot
\hline \hline
\endlastfoot
0 & \multicolumn{2}{c}{\bf Parallel beam incident from front ($+$Z-axis)}\\
  & \multicolumn{2}{c}{      RBEAM, UINC, VINC, WINC}\\
  & RBEAM   & radius of parallel beam in cm       \\
  &         & (defaults to the max radius of geometry)\\
  & UINC,VINC,WINC    & incident x,y,z-axis direction cosines \\
  %& VINC    & incident y-axis direction cosine\\
  %& WINC    & incident z-axis direction cosine\\
  &                & Note: (UINC,VINC,WINC) \\
  &                & get automatically normalized \\
  &                 &  defaults to (0.0,0.0,1.0) \\
\hline
1 & \multicolumn{2}{c}{\bf Point source on axis incident from front}\\
  & \multicolumn{2}{c}{      DISTZ, RBEAM, 0, 0}\\
  & DISTZ   & distance of the point source from the\\
  &                & front of the target in cm (DEFAULT 100.)\\
  & RBEAM   & radius of the beam at the front of the\\
  &                & target in cm (defaults to MAX radius) \\
\hline 
2 & \multicolumn{2}{c}{\bf Broad parallel beam from front ($+$Z-axis)} \\
  & \multicolumn{2}{c}{Basically a one dimensional calculation.} \\
  & \multicolumn{2}{c}{      0, 0, 0, 0} \\
 & & No input parameters needed.\\
\hline
3 & \multicolumn{2}{c}{\bf Internal uniform isotropically radiating disk of finite size}\\
  & \multicolumn{2}{c}{      RMINBM, RBEAM, ZSMIN, ZSMAX}\\
  & RMINBM  & inner radius of source region (inside\\
  &         & overall geometry)\\
  & RBEAM   & outer radius of source region (inside\\
  &         & overall geometry)\\
  & ZSMIN   & minimum z-coordinate of source \\
  & ZSMAX   & maximum z-coordinate of source \\
\hline
4 & \multicolumn{2}{c}{\bf Central axis depth-dose vs. beam radius}\\
  & \multicolumn{2}{c}{      RCAXIS, 0, 0, 0}\\
  & RCAXIS  & radius of central axis scoring zone (cm) \\
  &         & Radii of scoring zones will be treated as\\
  &         & beam radii.\\
\hline
10& \multicolumn{2}{c}{\bf Parallel beam incident from the side (along $+$Y-axis)}\\
  & \multicolumn{2}{c}{      XBEAM, ZBEAM, 0, 0}\\
  & XBEAM   & half-width of the rectangular beam in cm\\
  &                & (defaults to max radius)\\
  & ZBEAM   & half-height of the rectangular beam in cm\\
  &                & (defaults to max)\\
\hline
11&\multicolumn{2}{c}{\bf Point source incident from the side}\\
  &\multicolumn{2}{c}{      DISTRH, XBEAM, ZBEAM, 0}\\
  &DISTRH   & distance of the source from the middle\\
  &                & of the target in cm (defaults to 100.)\\
  &XBEAM    & half-width of the beam at the center of\\
  &                & the target in cm (defaults to max radius)\\
  &ZBEAM    & half-height of the beam at the center of\\
  &                & the target in cm (defaults to max)\\
\hline
12&\multicolumn{2}{c}{\bf Point source off axis}\\
  &\multicolumn{2}{c}{      DISTRH, DISTZ, 0, 0}\\
  &DISTRH   & distance of the point source off the Z-axis\\
  &DISTZ    & perpendicular distance of the point source\\
  &                & from the front face\\
  &          &if {\tt DISTZ}$>$0 \\
  &                & point is located in front of front face\\
  & & if 0 $>$ {\tt DISTZ} $>$ {\tt -(ZPLANE(NPLANE)-ZPLANE(1))}\\
  &                & point located between front and rear face\\
  & & {\tt DISTZ} $<$ {\tt -(ZPLANE(NPLANE)-ZPLANE(1))} \\
  &                & point located rear of rear plane \\
\hline
13&\multicolumn{2}{c}{\bf Parallel beam from any angle}\\
  &\multicolumn{2}{c}{      UINC, VINC, WINC, 0}\\
  & UINC    & incident x-axis direction cosine\\
  & VINC    & incident y-axis direction cosine\\
  & WINC    & incident z-axis direction cosine\\
  &         & Note: (UINC,VINC,WINC) get automatically\\
  &         & normalized. Default is (0.0,0.0,1.0)\\
\hline
14&\multicolumn{2}{c}{\bf Point source on axis incident from the front}\\
  &\multicolumn{2}{c}{\bf with all events inside RMINBM not followed} \\
  &\multicolumn{2}{c}{DISTZ, RBEAM, RMINBM, 0}\\
  & DISTZ   &  distance of the point source from the \\
  &         &  front of the target in cm (defaults to 100.)\\
  & RBEAM   &  radius of the beam at the front of the \\
  &         &  target in cm (defaults to max radius)\\
  & RMINBM  &  below this radius, all histories are terminated by\\
  &         &  the source routines by giving them zero weight. \\
  &         &  The HOWFAR routines must check for this.\\
\hline
15&\multicolumn{2}{c}{\bf Point source incident from any angle}\\
  &\multicolumn{2}{c}{DIST, ANGLE, 0, 0}\\
  & DIST    &  distance from the centre of the geometry to the\\
  &         &  point source (cm)\\
  & ANGLE   &  angle of rotation around the axis that is parallel \\
  &         &  to the x axis and passes trough the centre of the \\
  &         &  geometry (degrees).\\
  &         &  0 degrees corresponds to source above front face\\
  &         &  of geometry.\\
  &         &  Note: The point source must be outside the\\
  &         &  geometry.\\
\hline
16&\multicolumn{2}{c}{\bf Circular or rectangular isotropically-radiating source from any angle}\\
  &\multicolumn{2}{c}{DIST, ANGLE, TMP1, TMP2}\\
  & DIST    & distance from the centre of the geometry to the\\
  &         & centre of the source plane (cm)\\
  & ANGLE   & angle of rotation of source plane around the axis \\
  &         & that is parallel to the x axis and passes trough \\
  &         & the centre of the geometry (degrees). \\
  &         & Zero degrees corresponds to the\\
  &         & source incident on the front face of the\\
  &         & geometry.\\
  & TMP1    & if $>$ 0 and TMP2 $\leq$ 0: TMP1 = radius of\\
  &         & ~~~ source (cm)\\
  &         & if $\geq$ 0 and TMP2 $\geq$ 0: TMP1 = half-width\\
  &         & ~~~ of source in x direction (cm)\\
  & TMP2    & if $>$ 0 and TMP1 $\leq$ 0: TMP2 = radius of\\
  &         & ~~~ source (cm)\\
  &         & if $\geq$ 0 and TMP1 $\geq$ 0: TMP2 = half-width\\
  &         & ~~~ of source in y direction (cm)\\
  &         & Note: if TMP1 $\leq$ 0 and TMP2 $\leq$ 0, then this\\
  &         & becomes a point source incident from any angle,\\
  &         & identical to source 12 and 15.\\
\hline
17&\multicolumn{2}{c}{\bf Point source on-axis incident from the front (square collimation)}\\
&\multicolumn{2}{c}{DISTZ, XBEAM, YBEAM, 0}\\
& DISTZ & distance of source from front of geometry (cm).\\
&& Defaults to 100 cm.\\
& XBEAM & half-width in X-direction at front of geometry (cm).\\
& YBEAM & half-width in Y-direction at front of geometry (cm).\\
\hline
20&\multicolumn{2}{c}{\bf Radial distribution input}\\
  & \multicolumn{2}{c}{This source has no SOURCE PARAMETERS= inputs}\\
  & \multicolumn{2}{l}{\bf \underline {Additional inputs:}}\\
  & MODEIN= Local & if radial distribution is to be input\\
  &               & as part of the .egsinp file\\
  & MODEIN= External& if the distribution is to be input\\
  &               & via an external file\\
  &if MODEIN = Local &  \\
  & NRDIST       & Number of radial bins in distribution histogram\\
  & RDISTF       & top of radial bin should be NRDIST values \\
  & RPDF         & Probability of initial particle being in this bin.\\
  &              & Probability doesn't need to be normalized\\
  &              & but it should be in units cm$^{-2}$ \\
  &              & Should be values for 1 to NRDIST.\\
  & RDIST IOUTSP= None & No distribution data in output summary\\
  & ~~~~~~~~~~~~~~~~~~~~~ = Include & include distribution data output summary\\
  &if MODEIN = External & RDIST FILENAME \\
  & RDIST FILENAME     & filename(with ext) contains  distribution info\\
  &                   & in the same format as described above.\\
  & RDIST IOUTSP & See above\\
\hline
21&\multicolumn{2}{c}{\bf Full beam phase space data, incident on front face}\\
  &\multicolumn{2}{c}{IMODE, NRCYCL, IPARALLEL, PARNUM}\\
  & IMODE & set to 0 for 7 variables/record: X,Y,U,V,E,WT,LATCH \\
  &  & set to 2 for 8 variables/record: the above + ZLAST \\
  & NRCYCL  & no. of times to reuse each particle before\\
  &         & moving on to the next one.  Thus, each\\
  &         & particle is used a total of NRCYCL+1 times.\\
  &         &  Use of NRCYCL is essential for accurate\\
  &         & statistics when NCASE $>$ no. of particles\\
  &         & in the phase space file.  If set $\leq$0,\\
  &         & then NRCYCL is automatically calculated to\\
  &         & use the entire phase space file with no\\
  &         & restarts based on NCASE, incident particle\\
  &         & type, and the number of particles with the\\
  &         & appropriate charge in the phase\\
  &         & space file.  Note that the automatic\\
  &         & calculation of NRCYCL is not done if\\
  &         & INCIDENT PARTICLE= positron.\\
  &         & Also, the automatically-calculated value\\
  &         & of NRCYCL does not take into account\\
  &         & particles rejected because they miss the\\
  &         & geometry or because they have crossed the\\
  &         & phase space plane multiple times.\\
  & IPARALLEL & set $>$1 if you are splitting the\\
  &         & simulation into IPARALLEL jobs.  IPARALLEL\\
  &         & is used with PARNUM (see below) to partition\\
  &         & a phase space source into IPARALLEL equal\\
  &         & parts.\\
  & PARNUM  & For each of the IPARALLEL parallel jobs,\\
  &         & PARNUM should have a different integer value\\
  &         & in the range\\
  &         & range 1$\geq$PARNUM$\leq$PARALLEL.  The\\
  &         & partition of the phase space source that is\\
  &         & used for a job is then given by:\\
  & \multicolumn{2}{c}{(PARNUM-1)*(NCASE\_PHSP/IPARALLEL)$<$NPHSPN$\leq$}\\
  & \multicolumn{2}{c}{~~~~(PARNUM)*(NCASE\_PHSP/IPARALLEL)}\\
  &         &  where NCASE\_PHSP is the total number of\\
  &         & particles in the phsp source and NPHSPN\\
  &         & is the particle no. chosen.  Note that use of\\
  &         & IPARALLEL and PARNUM is not necessary if you are\\
  &         & using the built-in parallel processing functionality\\
  &         & of these codes.\\
  & \multicolumn{2}{l}{Note: IPARALLEL and PARNUM are not used with}\\
  & \multicolumn{2}{l}{the built-in EGSnrc\_MP parallel processing}\\
  & \multicolumn{2}{l}{functionality.}\\
  & \multicolumn{2}{l}{\bf \underline {Additional inputs:}}\\
  & FILSPC=  & Filename (with ext) containing the phase space\\
  &         & info (max. 80 characters) assigned  to unit 42.\\
\hline
22&\multicolumn{2}{c}{\bf Full beam phase space data, incident from any angle/position}\\
  &\multicolumn{2}{c}{IMODE, DIST, ANGLE, ZOFFSET, NRCYCL,}\\
  &\multicolumn{2}{c}{IPARALLEL, PARNUM, XOFFSET, YOFFSET}\\
  & IMODE & set to 0 for 7 variables/record: X,Y,U,V,E,WT,LATCH \\
  &  & set to 2 for 8 variables/record: the above + ZLAST \\
  & DIST & distance from source plane to the point\\
  &      & (x,y,z)=(0,0,ZOFFSET) in cm.  Defined so that,\\
  &      &  when ANGLE=0, DIST is in the -z direction.\\
  & ANGLE & angle of rotation about the Z axis (degrees).  \\ 
  &       &  ANGLE = 0 means particles are incident in the \\
  &       & +ve z direction.\\
  & ZOFFSET & defines the z offset of point from which DIST\\
  &         &  is measured. If $|$ZOFFSET$|$ $>$ 1e4, then it\\
  &         & defaults to the centre of the geometry.\\
  & NRCYCL & See source 21 above.\\
  & IPARALLEL & See source 21 above.\\
  & PARNUM & See source 21 above.\\
  & XOFFSET & X offset of source plane.\\
  & YOFFSET & Y offset of source plane.  X and Y offsets are\\
  &         & applied before any rotations by ANGLE.\\ 
& \multicolumn{2}{l}{\bf \underline {Additional inputs:}}\\
  & FILSPC=  & Filename (with ext) containing the phase space\\
  &         & info (max. 80 characters) assigned  to unit 42.\\
\hline
23&\multicolumn{2}{c}{\bf Full BEAM simulation, incident from any angle/position}\\
  &\multicolumn{2}{c}{DIST, ANGLE, ZOFFSET, XOFFSET, YOFFSET}\\
  & DIST & See source 22 above.  Note that the source plane is\\
  &      & the phase space scoring plane in the BEAM simulation\\
  &      & being used as a source.  See below for more details.\\ 
  & ANGLE & See source 22 above.\\
  & ZOFFSET & See source 22 above.\\
  & XOFFSET & See source 22 above.\\
  & YOFFSET & See source 22 above.\\
& \multicolumn{2}{l}{\bf \underline {Additional inputs:}}\\
  & BEAM CODE= & The name of the BEAM accelerator code\\
  &            & being used as a source, including the {\tt BEAM\_}\\
  &            & prefix (i.e. {\tt BEAM\_accelname}).  This code must\\
  &            & have been compiled as a shared library (see the BEAMnrc\\
  &            & manual for information on how to do this) and exist\\
  &            & as {\tt BEAM\_accelname.so} (for Linux/Unix) or\\
  &            & {\tt libBEAM\_accelname.dll} (for Windows) in your\\
  &            & {\tt \$EGS\_HOME/bin/config} directory.\\ 
  & INPUT FILE= & The name of the input file used for the accelerator\\
  &             & (no {\tt .egsinp} extension).  This file must exist\\
  &             & in your {\tt \$EGS\_HOME/BEAM\_accelname} directory and\\
  &             & must specify output of a phase space file at one scoring\\
  &             & plane.  Instead of being written to a phase space file,\\
  &             & particles are extracted and used as source particles upon\\
  &             & crossing this plane.\\
  & PEGS FILE= & The name of the pegs4 data set to be used in the BEAM\\
  &            & simulation (no {\tt .pegs4dat} extension).  The pegs4 data\\
  &            & must exist in {\tt \$HEN\_HOUSE/pegs4/data} or in your\\
  &            & {\tt \$EGS\_HOME/pegs4/data} directory.\\
  & WEIGHT WINDOW= & Set to MIN\_WEIGHT\_23, MAX\_WEIGHT\_23, where\\
  &            & MIN\_WEIGHT\_23 = min. weight of source particles to use\\
  &            & (defaults to -1E30) and MAX\_WEIGHT\_23 = max. weight of\\
  &            & source particles to use (defaults to 1E30).\\
  &&\\
  & \multicolumn{2}{l}{Note that the Z-position of the scoring plane, where incident particle}\\
  & \multicolumn{2}{l}{data is collected, in the BEAM simulation is not passed to the RZ code.}\\
  & \multicolumn{2}{l}{Thus, similar to a phase space source, the incident position of the source}\\
  & \multicolumn{2}{l}{plane, defined by {\tt DIST}, {\tt ANGLE}, {\tt ZOFFSET}, {\tt XOFFSET}}\\
  & \multicolumn{2}{l}{and {\tt YOFFSET}, is independent of the BEAM coordinate system.}\\
\hline
\label{tab:srcrz}
\end{longtable}
