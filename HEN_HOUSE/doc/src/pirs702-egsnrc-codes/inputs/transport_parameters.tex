
%%%%%%%%%%%%%%%%%%%%%%%%%%%%%%%%%%%%%%%%%%%%%%%%%%%%%%%%%%%%%%%%%%%%%%%%%%%%%%%
%
%  EGSnrc user codes transport parameters
%  Copyright (C) 2015 National Research Council Canada
%
%  This file is part of EGSnrc.
%
%  EGSnrc is free software: you can redistribute it and/or modify it under
%  the terms of the GNU Affero General Public License as published by the
%  Free Software Foundation, either version 3 of the License, or (at your
%  option) any later version.
%
%  EGSnrc is distributed in the hope that it will be useful, but WITHOUT ANY
%  WARRANTY; without even the implied warranty of MERCHANTABILITY or FITNESS
%  FOR A PARTICULAR PURPOSE.  See the GNU Affero General Public License for
%  more details.
%
%  You should have received a copy of the GNU Affero General Public License
%  along with EGSnrc. If not, see <http://www.gnu.org/licenses/>.
%
%%%%%%%%%%%%%%%%%%%%%%%%%%%%%%%%%%%%%%%%%%%%%%%%%%%%%%%%%%%%%%%%%%%%%%%%%%%%%%%
%
%  Authors:         Blake Walters, 2009
%
%  Contributors:    Frederic Tessier
%                   Ernesto Mainegra-Hing
%
%%%%%%%%%%%%%%%%%%%%%%%%%%%%%%%%%%%%%%%%%%%%%%%%%%%%%%%%%%%%%%%%%%%%%%%%%%%%%%%


\index{ECUT}
\begin{verbatim}
       Global ECUT=     Global (in all regions) electron transport cut
                        off energy (in MeV). If this imput is missing,
                        AE(medium) will be used.
                        [ ECUT ]
\end{verbatim}
\index{PCUT}
\begin{verbatim}
       Global PCUT=     Global (in all regions) photon transport cut
                        off energy (in MeV). If this imput is missing,
                        AP(medium) will be used.
                        [ PCUT ]
\end{verbatim}
\index{SMAX}
\begin{verbatim}
       Global SMAX=     Global (in all regions) maximum step-size
                        restriction for electron transport (in cm).
                        If missing, no geometrical step-size restrictions
                        will be employed. Note that if you use the default
                        EGSnrc electron-step algorithm, no SMAX-restriction
                        is necessary. Option is useful for transport in low
                        density materials (air) when PRESTA behaviour is
                        turned on (see below)
                        [ SMAXIR ]
\end{verbatim}
\index{ESTEPE}
\begin{verbatim}
       ESTEPE=          Maximum fractional energy loss per step.
                        Note that this is a global option only, no
                        region-by-region setting is possible. If missing,
                        the defualt is 0.25 (25%).
                        [ ESTEPE ]
\end{verbatim}
\index{XImax}
\begin{verbatim}
       XImax=           Maximum first elastic scattering moment per step.
                        Default is 0.5, NEVER use value greater than 1 as
                        this is beyond the range of MS data available.
                        [ XIMAX ]
\end{verbatim}
\index{boundary crossing algorithm}
\index{bca\_algorithm}
\index{exact\_bca}
\index{transport\_algorithm}
\begin{verbatim}
       Boundary crossing algorithm=
                        There are two selections possible: EXACT, means
                        the algorithm will cross boundaries in a single
                        scattering (SS) mode, the distance from a boundary
                        at which the transition to SS mode is made is
                        determined by 'Skin depth for BCA' (see below).
                        The second option is PRESTA-I, if selected boundaries
                        will be crossed a la PRESTA, i.e. with lateral
                        correlations turned off and MS forced at boundaries.
                        Default is EXACT.
                        [ bca_algorithm, exact_bca ]
\end{verbatim}
\index{skin depth for BCA}\index{exact\_bca}
\index{skindepth\_for\_bca}
\begin{verbatim}
       Skin depth for BCA=
                        Determines the distance from a boundary (in elastic
                        MFP) at which the algorithm will go into single
                        scattering mode (if EXACT boundary crossing) or
                        swith off lateral correlations (if PRESTA-I boundary
                        crossing). Default value is 3 for EXACT or
                        exp(BLCMIN)/BLCMIN for PRESTA-I (see the PRESTA paper
                        for a definition of BLCMIN). Note that if you choose
                        EXACT boundary crossing and set Skin depth for BCA
                        to a very large number (e.g. 1e10), the entire
                        calculation will be in SS mode. If you choose
                        PRESTA-I boundary crossing and make Skin depth for BCA
                        large, you will get default EGS4 behavious (no PRESTA)
                        [ skindepth_for_bca ]
\end{verbatim}
\index{electron step algorithm}
\begin{verbatim}
       Electron-step algorithm=
                        PRESTA-II (the default), the name is
                        used for historical reasons
                        or PRESTA-I
                        Determines the algorithm used to take into account
                        lateral and longitudinal correlations in a
                        condensed history step.
                        [ transport_algorithm ]
\end{verbatim}
\index{spin effects}
\index{spin\_effects}
\begin{verbatim}
       Spin effects=    Off, On, default is On
                        Turns off/on spin effects for electron elastic
                        scattering. Spin On is ABSOLUTELY necessary for
                        good backscattering calculations. Will make a
                        difference even in `well conditioned' situations
                        (e.g. depth dose curves for RTP energy range
                        electrons).
                        [ spin_effects ]
\end{verbatim}
\index{brems angular sampling}
\index{IBRDST}
\begin{verbatim}
       Brems angular sampling= Simple, KM, default is KM
                        If Simple, use only the leading term of the Koch-Motz
                        distribution to determine the emission angle of
                        bremsstrahlung photons. If On, complete
                        modified Koch-Motz 2BS is used (modifications
                        concern proper handling of kinematics at low energies,
                        makes 2BS almost the same as 2BN at low energies).
                        [ IBRDST ]
\end{verbatim}
\index{brems cross section}
\index{ibr\_nist}
\begin{verbatim}
       Brems cross sections= BH, NIST, NRC default is BH
                        If BH is selected, the Bethe-Heitler bremsstrahlung
                        cross sections (Coulomb corrected above 50 MeV)
                        will be used. If NIST is selected, the NIST brems
                        cross section data base (which is the basis for
                        the ICRU radiative stopping powers) will be employed.
                        Differences are negligible for E > ,say, 10 MeV,
                        but significant in the keV energy range. If NRC is
                        selected, the NRC brems cross-section data base will
                        be used, which is a version of the NIST data base
                        with corrected electron-electron brems contributions
                        (corrections to the NIST data is typically only
                        significant for low values of the atomic number Z
                        and for k/T < 0.005).
                        [ ibr_nist ]
\end{verbatim}
\index{triplet production}
\index{itriplet}
\begin{verbatim}
       Triplet production= On or Off (default).  Turns on/off simulation
                        of triplet production.  If On, then Borsellino's
                        first Born approximation is used to sample triplet
                        events based on the triplet cross-section data.
                        [ itriplet ]
\end{verbatim}
\index{bound Compton scattering}
\index{IBCMP}
\begin{verbatim}
       Bound Compton scattering=  On, Off, Simple or norej (default)
                        If Off, Compton scattering will be treated with
                        Klein-Nishina, with On Compton scattering is
                        treated in the Impulse approximation.
                        With Simple, the impulse approximation incoherent
                        scattering function will be used (i.e., no Doppler
                        broadenning). With norej the actual total bound
                        Compton cross section is used and there are no
                        rejections at run time.
                        Make sure to turn on for low energy applications,
                        not necessary above, say, 1 MeV.
                        [ IBCMP ]
\end{verbatim}
\index{radiative Compton corrections}
\index{radc\_flag}
\begin{verbatim}
       Radiative Compton corrections= On or Off (default). If on, then
                        include radiative corrections for Compton scattering.
                        Equations are based on original Brown & Feynman
                        equations (Phys. Rev. 85, p 231--1952).  Requires
                        a change to the user codes Makefile to include
                        $(EGS_SOURCEDIR)rad_compton1.mortran in the
                        SOURCES (just before $(EGS_SOURCEDIR)egsnrc.mortran).
                        [ radc_flag ]
\end{verbatim}
\index{electron impact ionization}
\index{eii\_flag}
\index{eii\_xfile}
\begin{verbatim}
       Electron Impact Ionization= Off (default), On, casnati, kolbenstvedt,
                        gryzinski or penelope.  If set to On or ik, then
                        use Kawrakow's theory to derive EII cross-sections.
                        If set to casnati, then use the cross-sections of
                        Casnati (from file ($HEN_HOUSE/data/eii_casnati.data).
                        Similar for kolbenstvedt, gryzinski and penelope.
                        This is only of interest in kV X-ray calculations.
                        Note that the user can supply their own EII
                        cross-section data as well. The requirement is that
                        the file eii_suffix.data exists in the $HEN_HOUSE/data
                        directory, where suffix is the name specified.
                        Entry is case-sensitive except for Off, On or ik.
                        [ eii_flag, eii_xfile ]
\end{verbatim}
\index{pair angular sampling}
\index{IPRDST}
\begin{verbatim}
       Pair angular sampling= Off, Simple, KM.
                        If off, pairs are set in motion at an angle m/E
                        relative to the photon direction (m is electron rest
                        energy, E the photon energy). Simple turns on
                        the leading term of the angular distribution
                        (this is sufficient for most applications),
                        KM (comes from Koch and Motz) turns on using 2BS
                        from the article by Koch and Motz.  Uniform
                        Default is Simple, make sure you always use
                        Simple or KM
                        [ IPRDST ]
\end{verbatim}
\index{pair cross sections}
\index{pair\_nrc}
\begin{verbatim}
       Pair cross sections= BH (default) or NRC.  If set to BH, then use
                        Bethe-Heitler pair production cross-sections.  If set
                        to NRC, then use NRC pair production cross-sections
                        (in file $HEN_HOUSE/data/pair_nrc1.data).  Only
                        of interest at low energies, where the NRC cross-
                        sections take into account the assymmetry in the
                        positron-electron energy distribution.
                        [ pair_nrc ]
\end{verbatim}
\index{photon cross sections}
\index{photon\_xsections}
\begin{verbatim}
       Photon cross sections= Photon cross-section data.  Current options are
                        si (Storm-Israel--the default), epdl (Evaluated Photon
                        Data Library), xcom, and pegs4.  Allows the use of photon
                        cross-sections other than from the PEGS4 file unless
                        the pegs4 option is specified.
                        Note that the user can supply their own cross-section
                        data as well.  The requirement is that the files
                        photon_xsections_photo.data,
                        photon_xsections_pair.data,
                        photon_xsections_triplet.data, and
                        photon_xsections_rayleigh.data exist in the
                        $HEN_HOUSE/data directory, where photon_xsections
                        is the name specified.
                        Entry is case-sensitive except for the pegs4 option.
                        [ photon_xsections ]
\end{verbatim}
\index{photon cross sections!output}
\index{xsec\_out}
\begin{verbatim}
       Photon cross-sections output= Off (default) or On.  If On, then
                        a file $EGS_HOME/user_code/inputfile.xsections is
                        output containing photon cross-section data used.
                        [ xsec_out ]
\end{verbatim}
\index{compton cross sections}
\index{comp\_xsections}
\begin{verbatim}
       Compton cross sections= Bound Compton cross-section data.  User-
                        supplied bound Compton cross-sections in the file
                        $HEN_HOUSE/data/comp_xsections_compton.data, where
                        comp_xsections is the name supplied for this input.
                        This is only used if Bound Compton scattering= Simple
                        and is not available on a region-by-region basis
                        (see below).  The default file (ie in the absence
                        of any user-supplied data) is compton_sigma.data.
                        [ comp_xsections ]
\end{verbatim}
\index{Rayleigh scattering}
\index{Rayleigh scattering!custom form factors}
\index{IRAYLR}
\begin{verbatim}
       Rayleigh scattering= Off, On, custom
                        If On, turned on coherent (Rayleigh) scattering.
                        Default is Off. Should be turned on for low energy
                        applications. If custom, user must provide media names
                        and form factor files for each desired medium. The
                        rest of the media use the default atomic form factors.
                        Not set to On by default for historical reasons since
                        a PEGS4 data set is not required anymore.
                        [ IRAYLR ]
\end{verbatim}
\index{iray\_ff\_media}
\begin{verbatim}
       ff media names = A list of media names (must match media found in
                        PEGS4 data file) for which the user is going to
                        provide custom Rayleigh form factor data.
                        [ iray_ff_media($MXMED) ]
\end{verbatim}
\index{iray\_ff\_file}
\begin{verbatim}
       ff file names = A list of names of files containing the Rayleigh
                       form factor data for the media specified by
                       the ff media names = input above.  Full directory
                       paths must be given for all files, and for each medium
                       specified, iray_ff_media(i), there must be a
                       corresponding file name, iray_ff_file(i).  For
                       example files, see the directory
                       $HEN_HOUSE/data/molecular_form_factors.
                       [ iray_ff_file($MXMED) ]
\end{verbatim}
\index{photoelectron angular sampling}
\index{IPHTER}
\begin{verbatim}
       Photoelectron angular sampling= Off or On
                        If Off, photo-electrons get the direction of the
                        `mother' photon, with On, Sauter's furmula is
                        used (which is, striktly speaking, valid only for
                        K-shell photo-absorption).
                        If the user has a better approach, replace the macro
                            $SELECT-PHOTOELECTRON-DIRECTION;
                        The only application that
                        I encountered until now where this option made a
                        small difference was a big ion chamber (cavity size
                        comparable with electron range) with high-Z walls
                        in a low energy photon beam.
                        Default is On
                        [ IPHTER ]
\end{verbatim}
\index{atomic relaxations}
\index{IEDGFL}
\begin{verbatim}
       Atomic relaxations= Off, On
                        Default is On. The effect of using On is twofold:
                        - In photo-electric absorption events, the element
                          (if material is mixture) and the shell the photon
                          is interacting with are sampled from the appropriate
                          cross seections
                        - Shell vacancies created in photo-absorption events
                          are relaxed via emission of fluorescent X-Rays,
                          Auger and Koster-Cronig electrons.
                         Make sure to turn this option on for low energy
                         applications.
                         [ IEDGFL ]
\end{verbatim}

\noindent
Atomic relaxations, Rayleigh scattering, Photoelectron angular sampling
and Bound Compton scattering can also be turned On/Off on a
region-by-region basis. An example for Atomic relaxations on a region-
by-region basis is:

\begin{verbatim}
       Atomic relaxations= On in Regions   or
       Atomic relaxations= Off in regions
\end{verbatim}

Then define the regions in which you want the feature to be turned on:

\begin{verbatim}
       Bound Compton start region=
       Bound Compton stop region=
                or
       Rayleigh start region=
       Rayleigh stop region=
                or
       Relaxations start region=
       Relaxations stop region=
                or
       PE sampling start region=
       PE sampling stop region=
\end{verbatim}
each followed by a list of one or more start and stop regions
separated by commas. Example:
\begin{verbatim}
        Atomic relaxations= On in Regions
        Relaxations start region=  1, 40
        Relaxations stop region=  10, 99
\end{verbatim}
will first turn off relaxations everywhere and
then turn on in regions 1-10 and 40-99.
Note that the input is checked against minimum and maximum
region number and ignored if
\verb+start region < 1+ or \verb+stop_region > $MXREG+ or
\verb+start region > stop region+.

\verb+ECUT+, \verb+PCUT+ and \verb+SMAX+ can also be set on a
region-by-region basis. To do so, include in the input file
\begin{verbatim}
         Set XXXX=              f_value1, f_value2, ...
         Set XXXX start region= i_value1, i_value2, ...
         Set XXXX stop region=  j_value1, j_value2, ...
\end{verbatim}
where \verb+XXXX+ is \verb+ECUT+, \verb+PCUT+ or \verb+SMAX+,
\verb+f_value1+, \verb+f_value2+,...
are the desired values for \verb+XXXX+ and \verb+i_value_i+ and
\verb+j_value_i+ are the start and stop regions.
